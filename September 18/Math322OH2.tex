\documentclass[10.5pt]{article}
\usepackage{amsmath, amsfonts, amssymb,amsthm}
\usepackage{centernot}
\usepackage[includeheadfoot,margin=0.5in]{geometry} % For page dimensions
\usepackage{fancyhdr}
\usepackage{enumerate} % For custom lists

\fancyhf{}
\lhead{}
\rhead{}
\pagestyle{fancy}

% Page dimensions
\geometry{a4paper}

\theoremstyle{definition}
\newtheorem{pb}{}

% Commands:

\newcommand{\set}[1]{\{#1\}}
\newcommand{\abs}[1]{\lvert#1\rvert}
\newcommand{\norm}[1]{\lvert\lvert#1\rvert\rvert}
\newcommand{\tand}[1]{\text{ and }}
\newcommand{\tor}[1]{\text{ or }}

\title{\vspace{-2cm} Math 322 Office Hours Recap}
\date{September 18, 2024}
\begin{document}
    \maketitle

    \textbf{Semi-Group Homework Problem}

    \begin{enumerate}
        \item Jacobson 1.2.11 Show that in a group, the equations \(ax = b\) and \(ya = b\) are solveable. Conversely show that any semigroup with this property is a group.

        \textbf{Solution.} Since this is a homework problem, we didn't go over it in full detail. The forward direction is relatively simple. To show the converse,
        we need to show existence of identity and inverses. Identity must be shown first. Notice that the equation \(ax = a\) gives us a right identity for free,
        we want to show this is an identity in general. In order to do so we want to show that \(x\) is a identity on the entire semigroup, and is two sided. The only way
        we can introduce more information into our equation \(ax = a\) is by taking a product with some other equation, but since solveability gives us uniqueness, thsi should be enough.
        We let \(b \in G\) be arbitrary, and define \(c = xb\) (since we want to take a product where \(x\) is on the right side of \(a\) we must put it on the left side of \(b\)).
        This gives us that \(ab = (ax)b = a(xb) = ac\), so unique solveability of the equation \(ay = ab\) implies that \(b = c\), making \(x\) a left inverse for the entire
        set since \(b\) was arbitrary. Hopefully this gives some insight into how to finish the problem, as the solution to showing \(x\) is a right identity is almost identical,
        and existence of inverses is easier.
    \end{enumerate}

    \textbf{Linear Algebra and Equivalence Relation Exam Problem}
    \begin{enumerate}
        \item Consider \(X = \set{(v_1,v_2) \vert v_i \in \mathbb{R}^2}\), and equivalence relation \((\mathbf{v}_1,\mathbf{v}_2) \sim (\mathbf{u}_1,\mathbf{u}_2) \iff \exists A, \; A\mathbf{v}_i = \mathbf{u}_i\), where
        \(A\) is invertible. Describe the equivalence classes of \(X/\sim\).

        \textbf{Solution.} The equivalence classes are subspaces of \(\mathbb{R}^2\), proof being:
        \begin{align*}
            A\mathbf{v}_i = \mathbf{u}_i \iff A\begin{bmatrix} \mathbf{v}_1 \mathbf{v}_2 \end{bmatrix} = \begin{bmatrix} \mathbf{u}_1 \mathbf{u}_2 \end{bmatrix}
        \end{align*}
        But multiplying by an invertible matrix on the left is equivalent to taking row operations, so \((\mathbf{v}_1,\mathbf{v}_2) \sim (\mathbf{u}_1,\mathbf{u}_2)\) is equivalent to
        \begin{align*}
            \text{rowsp}\begin{bmatrix} \mathbf{v}_1 \mathbf{v}_2 \end{bmatrix} = \text{rowsp}\begin{bmatrix} \mathbf{u}_1 \mathbf{u}_2 \end{bmatrix}
        \end{align*}
        So two pairs are similar when they have the same rowspace. This means that the equivalence classes are all subspaces of \(\mathbb{R}^2\)
    \end{enumerate}

    \textbf{Herstein Challenge Problem}
    \begin{enumerate}
        \item Herstein 2.5.26 If an abelian group \(G\) has subgroups \(M, N\) of order \(m, n\) respectively, show that it has a subgroup of order \(\text{lcm}[m,n]\)
        
        There are two approaches for this problem, the first is to look at \(\# MN\) (which is a subgroup since \(G\) is abelian). \emph{hint if you are stuck}: use Herstein Theorem \(2.5.1\).

        The second is to prove and use \emph{Cauchy's Theorem}, that \(p \vert \# G \implies \exists g \in G, \; o(g) = p\). \emph{hint if you are stuck}: Use quotient groups and induction.


    \end{enumerate}
\end{document}