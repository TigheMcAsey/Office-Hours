\documentclass[10.5pt]{article}
\usepackage{amsmath, amsfonts, amssymb,amsthm}
\usepackage{centernot}
\usepackage[includeheadfoot,margin=0.5in]{geometry} % For page dimensions
\usepackage{fancyhdr}
\usepackage{enumerate} % For custom lists

\fancyhf{}
\lhead{}
\rhead{}
\pagestyle{fancy}

% Page dimensions
\geometry{a4paper}

\theoremstyle{definition}
\newtheorem{pb}{}

% Commands:

\newcommand{\set}[1]{\{#1\}}
\newcommand{\abs}[1]{\lvert#1\rvert}
\newcommand{\norm}[1]{\lvert\lvert#1\rvert\rvert}
\newcommand{\tand}[1]{\text{ and }}
\newcommand{\tor}[1]{\text{ or }}

\title{\vspace{-2cm} Math 322 Office Hours Recap}
\date{September 11, 2024}
\begin{document}
    \maketitle
    
    \textbf{Proofs Involving Quadratic Polynomials}

    \begin{enumerate}
        \item \(f: \mathbb{R} \to \mathbb{R}, \; f(x) = x^2 + ax + b\), is \(f\) injective? \newline
        \textbf{Solution.} No. To see this, notice that \(f(x)\) is a parabola, so if it is centered at \(x = x_0\), then \(f(x_0 + t) = f(x_0 - t), \forall t\).
        To prove that \(f\) is not injective, we can rewrite the function;
        \begin{align*}
            f(x) = x^2 + ax + b = x^2 + ax + \frac{a^2}{4} - \frac{a^2}{4} + b = \left(x+\frac{a}{2}\right)^2 + b - \frac{a^2}{4}
        \end{align*}
        This gives us our \(x_0 = -\frac{a}{2}\), so that
        \begin{align*}
            f\left(-\frac{a}{2} + 1\right) = \left(-\frac{a}{2}+\frac{a}{2} + 1\right)^2 + b - \frac{a^2}{4} = 1 + b - \frac{a^2}{4} = \left(-\frac{a}{2}+\frac{a}{2} - 1\right)^2 + b - \frac{a^2}{4}
            = f\left(-\frac{a}{2} - 1\right)
        \end{align*}

        \item Let \(f\) be the same as above, is it surjective? \newline
        \textbf{Solution.} No. To see this, recognize the graph of the parabola is bounded below.
        To prove that \(f\) is bounded below we once again rewrite it in the form \(\left(x+\frac{a}{2}\right)^2 + b - \frac{a^2}{4}\), then
        \begin{align*}
            \forall x, \; \left(x+\frac{a}{2}\right)^2 \geq 0 \implies f(x) \geq b - \frac{a^2}{4}
        \end{align*}
        So choose \(y = b - \frac{a^2}{4} - 1\), since \(y < b - \frac{a^2}{4} \leq f(x)\) the preimage of \(y\) is empty so that \(f\) is not surjective.

        \item Is \(f\) bounded above? \newline
        \textbf{Solution.} No. To see this, recognize that the graph of \(f\) is a parabola which is unbounded.
        To prove that \(f\) is unbounded, let \(M \in \mathbb{R}_{\geq 0}\), the direct way to approach this is to work backwards given the desired inequality,
        \begin{align*}
            x^2 + ax + b &\geq M \\
            x(x + a) &\geq M - b \; \text{Notice here we can assume that } x\geq \max\set{\abs{a},1}\\
            x + a &\geq M - b\\
            x &= \max\set{M-b+a,1,\abs{a}}
        \end{align*}
        With this choice of \(x\), we can finish the proof:
        \begin{align*}
            &x(x+a) \geq x + a = M - b \implies x^2 + ax + b \geq M
        \end{align*}
        Alternatively, we can use the same approach as before and write \(f(x) = \left(x+\frac{a}{2}\right)^2 + b - \frac{a^2}{4}\), and once again try to solve for \(x\),
        \begin{align*}
            \left(x+\frac{a}{2}\right)^2 + b - \frac{a^2}{4} &\geq M \\
            \left(x+\frac{a}{2}\right)^2 &\geq M - b + \frac{a^2}{4} \\
        \end{align*}
        Here we notice if the right hand side is negative, then we can choose any \(x\), so assume its positive,
        \begin{align*}
            x + \frac{a}{2} &\geq \sqrt{M - b + \frac{a^2}{4}} \\
            x &\geq \sqrt{M - b + \frac{a^2}{4}} - \frac{a}{2}
        \end{align*}
        So just choose \(x = \sqrt{M - b + \frac{a^2}{4}} - \frac{a}{2}\).
    \end{enumerate}

    \textbf{Remark.} The 'completing the square' trick, where we rewrite the parabola was highly useful in these proofs.
    Intuitively, we can think of this as a change of coordinates, or variable substitution \(x \to u\) where the parabola is centered around \(u\). The reason this
    is easier is because most of the behaviour we were interested in was reliant on the paint the parabola was centered around, so choosing convenient coordinates
    which allow us to more easily describe this behaviour makes our life much easier. Analogous techniques are common in math, where often there is a
    convenient choice of basis or generators which makes computations easier.
\end{document}